% Options for packages loaded elsewhere
% Options for packages loaded elsewhere
\PassOptionsToPackage{unicode}{hyperref}
\PassOptionsToPackage{hyphens}{url}
\PassOptionsToPackage{dvipsnames,svgnames,x11names}{xcolor}
%
\documentclass[
  letterpaper,
  DIV=11,
  numbers=noendperiod]{scrartcl}
\usepackage{xcolor}
\usepackage{amsmath,amssymb}
\setcounter{secnumdepth}{5}
\usepackage{iftex}
\ifPDFTeX
  \usepackage[T1]{fontenc}
  \usepackage[utf8]{inputenc}
  \usepackage{textcomp} % provide euro and other symbols
\else % if luatex or xetex
  \usepackage{unicode-math} % this also loads fontspec
  \defaultfontfeatures{Scale=MatchLowercase}
  \defaultfontfeatures[\rmfamily]{Ligatures=TeX,Scale=1}
\fi
\usepackage{lmodern}
\ifPDFTeX\else
  % xetex/luatex font selection
\fi
% Use upquote if available, for straight quotes in verbatim environments
\IfFileExists{upquote.sty}{\usepackage{upquote}}{}
\IfFileExists{microtype.sty}{% use microtype if available
  \usepackage[]{microtype}
  \UseMicrotypeSet[protrusion]{basicmath} % disable protrusion for tt fonts
}{}
\makeatletter
\@ifundefined{KOMAClassName}{% if non-KOMA class
  \IfFileExists{parskip.sty}{%
    \usepackage{parskip}
  }{% else
    \setlength{\parindent}{0pt}
    \setlength{\parskip}{6pt plus 2pt minus 1pt}}
}{% if KOMA class
  \KOMAoptions{parskip=half}}
\makeatother
% Make \paragraph and \subparagraph free-standing
\makeatletter
\ifx\paragraph\undefined\else
  \let\oldparagraph\paragraph
  \renewcommand{\paragraph}{
    \@ifstar
      \xxxParagraphStar
      \xxxParagraphNoStar
  }
  \newcommand{\xxxParagraphStar}[1]{\oldparagraph*{#1}\mbox{}}
  \newcommand{\xxxParagraphNoStar}[1]{\oldparagraph{#1}\mbox{}}
\fi
\ifx\subparagraph\undefined\else
  \let\oldsubparagraph\subparagraph
  \renewcommand{\subparagraph}{
    \@ifstar
      \xxxSubParagraphStar
      \xxxSubParagraphNoStar
  }
  \newcommand{\xxxSubParagraphStar}[1]{\oldsubparagraph*{#1}\mbox{}}
  \newcommand{\xxxSubParagraphNoStar}[1]{\oldsubparagraph{#1}\mbox{}}
\fi
\makeatother

\usepackage{color}
\usepackage{fancyvrb}
\newcommand{\VerbBar}{|}
\newcommand{\VERB}{\Verb[commandchars=\\\{\}]}
\DefineVerbatimEnvironment{Highlighting}{Verbatim}{commandchars=\\\{\}}
% Add ',fontsize=\small' for more characters per line
\usepackage{framed}
\definecolor{shadecolor}{RGB}{241,243,245}
\newenvironment{Shaded}{\begin{snugshade}}{\end{snugshade}}
\newcommand{\AlertTok}[1]{\textcolor[rgb]{0.68,0.00,0.00}{#1}}
\newcommand{\AnnotationTok}[1]{\textcolor[rgb]{0.37,0.37,0.37}{#1}}
\newcommand{\AttributeTok}[1]{\textcolor[rgb]{0.40,0.45,0.13}{#1}}
\newcommand{\BaseNTok}[1]{\textcolor[rgb]{0.68,0.00,0.00}{#1}}
\newcommand{\BuiltInTok}[1]{\textcolor[rgb]{0.00,0.23,0.31}{#1}}
\newcommand{\CharTok}[1]{\textcolor[rgb]{0.13,0.47,0.30}{#1}}
\newcommand{\CommentTok}[1]{\textcolor[rgb]{0.37,0.37,0.37}{#1}}
\newcommand{\CommentVarTok}[1]{\textcolor[rgb]{0.37,0.37,0.37}{\textit{#1}}}
\newcommand{\ConstantTok}[1]{\textcolor[rgb]{0.56,0.35,0.01}{#1}}
\newcommand{\ControlFlowTok}[1]{\textcolor[rgb]{0.00,0.23,0.31}{\textbf{#1}}}
\newcommand{\DataTypeTok}[1]{\textcolor[rgb]{0.68,0.00,0.00}{#1}}
\newcommand{\DecValTok}[1]{\textcolor[rgb]{0.68,0.00,0.00}{#1}}
\newcommand{\DocumentationTok}[1]{\textcolor[rgb]{0.37,0.37,0.37}{\textit{#1}}}
\newcommand{\ErrorTok}[1]{\textcolor[rgb]{0.68,0.00,0.00}{#1}}
\newcommand{\ExtensionTok}[1]{\textcolor[rgb]{0.00,0.23,0.31}{#1}}
\newcommand{\FloatTok}[1]{\textcolor[rgb]{0.68,0.00,0.00}{#1}}
\newcommand{\FunctionTok}[1]{\textcolor[rgb]{0.28,0.35,0.67}{#1}}
\newcommand{\ImportTok}[1]{\textcolor[rgb]{0.00,0.46,0.62}{#1}}
\newcommand{\InformationTok}[1]{\textcolor[rgb]{0.37,0.37,0.37}{#1}}
\newcommand{\KeywordTok}[1]{\textcolor[rgb]{0.00,0.23,0.31}{\textbf{#1}}}
\newcommand{\NormalTok}[1]{\textcolor[rgb]{0.00,0.23,0.31}{#1}}
\newcommand{\OperatorTok}[1]{\textcolor[rgb]{0.37,0.37,0.37}{#1}}
\newcommand{\OtherTok}[1]{\textcolor[rgb]{0.00,0.23,0.31}{#1}}
\newcommand{\PreprocessorTok}[1]{\textcolor[rgb]{0.68,0.00,0.00}{#1}}
\newcommand{\RegionMarkerTok}[1]{\textcolor[rgb]{0.00,0.23,0.31}{#1}}
\newcommand{\SpecialCharTok}[1]{\textcolor[rgb]{0.37,0.37,0.37}{#1}}
\newcommand{\SpecialStringTok}[1]{\textcolor[rgb]{0.13,0.47,0.30}{#1}}
\newcommand{\StringTok}[1]{\textcolor[rgb]{0.13,0.47,0.30}{#1}}
\newcommand{\VariableTok}[1]{\textcolor[rgb]{0.07,0.07,0.07}{#1}}
\newcommand{\VerbatimStringTok}[1]{\textcolor[rgb]{0.13,0.47,0.30}{#1}}
\newcommand{\WarningTok}[1]{\textcolor[rgb]{0.37,0.37,0.37}{\textit{#1}}}

\usepackage{longtable,booktabs,array}
\usepackage{calc} % for calculating minipage widths
% Correct order of tables after \paragraph or \subparagraph
\usepackage{etoolbox}
\makeatletter
\patchcmd\longtable{\par}{\if@noskipsec\mbox{}\fi\par}{}{}
\makeatother
% Allow footnotes in longtable head/foot
\IfFileExists{footnotehyper.sty}{\usepackage{footnotehyper}}{\usepackage{footnote}}
\makesavenoteenv{longtable}
\usepackage{graphicx}
\makeatletter
\newsavebox\pandoc@box
\newcommand*\pandocbounded[1]{% scales image to fit in text height/width
  \sbox\pandoc@box{#1}%
  \Gscale@div\@tempa{\textheight}{\dimexpr\ht\pandoc@box+\dp\pandoc@box\relax}%
  \Gscale@div\@tempb{\linewidth}{\wd\pandoc@box}%
  \ifdim\@tempb\p@<\@tempa\p@\let\@tempa\@tempb\fi% select the smaller of both
  \ifdim\@tempa\p@<\p@\scalebox{\@tempa}{\usebox\pandoc@box}%
  \else\usebox{\pandoc@box}%
  \fi%
}
% Set default figure placement to htbp
\def\fps@figure{htbp}
\makeatother





\setlength{\emergencystretch}{3em} % prevent overfull lines

\providecommand{\tightlist}{%
  \setlength{\itemsep}{0pt}\setlength{\parskip}{0pt}}



 


\KOMAoption{captions}{tableheading}
\makeatletter
\@ifpackageloaded{caption}{}{\usepackage{caption}}
\AtBeginDocument{%
\ifdefined\contentsname
  \renewcommand*\contentsname{Table of contents}
\else
  \newcommand\contentsname{Table of contents}
\fi
\ifdefined\listfigurename
  \renewcommand*\listfigurename{List of Figures}
\else
  \newcommand\listfigurename{List of Figures}
\fi
\ifdefined\listtablename
  \renewcommand*\listtablename{List of Tables}
\else
  \newcommand\listtablename{List of Tables}
\fi
\ifdefined\figurename
  \renewcommand*\figurename{Figure}
\else
  \newcommand\figurename{Figure}
\fi
\ifdefined\tablename
  \renewcommand*\tablename{Table}
\else
  \newcommand\tablename{Table}
\fi
}
\@ifpackageloaded{float}{}{\usepackage{float}}
\floatstyle{ruled}
\@ifundefined{c@chapter}{\newfloat{codelisting}{h}{lop}}{\newfloat{codelisting}{h}{lop}[chapter]}
\floatname{codelisting}{Listing}
\newcommand*\listoflistings{\listof{codelisting}{List of Listings}}
\makeatother
\makeatletter
\makeatother
\makeatletter
\@ifpackageloaded{caption}{}{\usepackage{caption}}
\@ifpackageloaded{subcaption}{}{\usepackage{subcaption}}
\makeatother
\usepackage{bookmark}
\IfFileExists{xurl.sty}{\usepackage{xurl}}{} % add URL line breaks if available
\urlstyle{same}
\hypersetup{
  pdftitle={Assignment 1 --- Matching Pennies: WSLS vs k-ToM-inspired belief learning},
  pdfauthor={Amalie Overgaard Stevnhøj Pedersen, Gréta Harsányi, Mads Munch Mikkelsen, Ramona Tanović},
  colorlinks=true,
  linkcolor={blue},
  filecolor={Maroon},
  citecolor={Blue},
  urlcolor={Blue},
  pdfcreator={LaTeX via pandoc}}


\title{Assignment 1 --- Matching Pennies: WSLS vs k-ToM-inspired belief
learning}
\author{Amalie Overgaard Stevnhøj Pedersen, Gréta Harsányi, Mads Munch
Mikkelsen, Ramona Tanović}
\date{}
\begin{document}
\maketitle

\renewcommand*\contentsname{Table of contents}
{
\hypersetup{linkcolor=}
\setcounter{tocdepth}{3}
\tableofcontents
}

\section{What this assignment does}\label{what-this-assignment-does}

This document (i) describes two cognitively plausible strategies for
repeated Matching Pennies, (ii) formalises them as executable rule-based
models (diagrams + equations), and (iii) evaluates their behaviour in
simulation. Finally, it treats the strategies as mechanistic statistical
models and checks whether they are distinguishable when fit to data
(here: synthetic data).

Repository (all code, simulations, figures, reproducibility):

\begin{itemize}
\tightlist
\item
  https://github.com/ramona-tanovic/ACM.git
\end{itemize}

The relevant folder is \texttt{assignment1/}.

\section{Data + figures used in this
write-up}\label{data-figures-used-in-this-write-up}

This document reads the CSV outputs produced by
\texttt{assignment1/run\_all.R} and inserts the key numbers directly
into the text (so the narrative stays consistent if you re-run the
pipeline).

\section{The game and protocol}\label{the-game-and-protocol}

Matching Pennies is a two-player, zero-sum game. On each trial both
players choose an action (Left/Right). One player is the
\textbf{Matcher} (wins if actions match), the other is the
\textbf{Mismatcher} (wins if actions differ).

In the repeated task, roles swap halfway through the block. That swap is
cognitively informative: it tests whether a strategy merely reacts to
reinforcement history or whether it represents the \emph{current
contingency} (what counts as ``good'' right now).

In the simulations used for the figures in this document, there are (T =
)100 trials per match, and the role swap is at trial (50).

\includegraphics[width=8.87in,height=2.31in]{assignment1_files/figure-latex/mermaid-figure-1.png}

\section{Strategy 1: Win--Stay / Lose--Shift
(WSLS)}\label{strategy-1-winstay-loseshift-wsls}

\subsection{Intuition}\label{intuition}

WSLS is a classic reinforcement heuristic:

\begin{itemize}
\tightlist
\item
  If the previous action led to a win, repeat it.
\item
  If it led to a loss, switch.
\end{itemize}

It is cognitively plausible because it only requires one-step memory and
a simple if/else rule. It does not need a representation of the opponent
or of the game structure.

\subsection{Formalisation}\label{formalisation}

Let (a\_\{t-1\}) be the previous action and (w\_\{t-1\}\in\{0,1\})
indicate whether the agent won last trial.

Parameters: - (p\_\{\text{repeat|win}\}): repeat after win -
(p\_\{\text{repeat|loss}\}): repeat after loss (often low in WSLS) -
(\ell): lapse/noise: with probability (\ell), choose randomly

\includegraphics[width=7.39in,height=7.95in]{assignment1_files/figure-latex/mermaid-figure-3.png}

\subsection{Cognitive constraints
(WSLS)}\label{cognitive-constraints-wsls}

WSLS is a bounded-memory, low-computation control policy. It is a
plausible ``default heuristic'' when agents cannot or do not build an
internal model of the opponent or task.

The key limitation is that it cannot condition explicitly on the
match/mismatch goal. It only changes by accumulating reinforcement
feedback, so any re-mapping after the role swap is indirect and may lag.

\section{Strategy 2: k-ToM-inspired belief learning
(kToM)}\label{strategy-2-k-tom-inspired-belief-learning-ktom}

\subsection{Intuition}\label{intuition-1}

This strategy maintains a simple belief about the opponent's action
tendency and updates that belief from observations. It is
``k-ToM-inspired'' in that the agent behaves as if the opponent has a
stable (possibly biased) policy and tries to exploit it. The belief is
updated incrementally with a learning rate, and action selection is
conditioned on the current role (Matcher vs Mismatcher).

It remains cognitively constrained: rather than storing the full
history, it compresses experience into a single belief state and updates
incrementally.

\subsection{Formalisation}\label{formalisation-1}

Let (p\_t = P(b\_t = 1)) be the belief that the opponent will choose
action 1.

Belief update: {[} p\_\{t+1\} = (1-\alpha),p\_t + \alpha,b\_t, {]} where
(0\textless{}\alpha\textless1) controls recency weighting.

Choice uses an inverse temperature (\beta) (sharper exploitation for
larger (\beta)) and a lapse (\ell). Crucially, action selection is
conditioned on role: - Matcher: choose the action that matches the
predicted opponent action - Mismatcher: choose the opposite action

\includegraphics[width=6.1in,height=10.12in]{assignment1_files/figure-latex/mermaid-figure-2.png}

\subsection{Cognitive constraints
(kToM)}\label{cognitive-constraints-ktom}

This model adds a minimal internal state (belief (p\_t)). That increases
cognitive demands relative to WSLS but still reflects bounded resources:
the entire history is summarised by one scalar belief, updated with a
single learning-rate parameter.

Because the goal is represented explicitly (match vs mismatch), the
model can re-map actions when roles swap without waiting for long
reinforcement transients.

\section{Implementation logic: executable rule-based
models}\label{implementation-logic-executable-rule-based-models}

Both strategies are implemented as ``model objects'': a parameter vector
plus a choice rule and an update rule. This is the core modelling move
in rule-based cognitive modelling: the hypothesis is expressed as an
executable procedure that generates behaviour, not just as a verbal
description.

\section{Simulation design}\label{simulation-design}

We run a tournament of pairings (each strategy against itself and
against the other). For each pairing we simulate (600) matches. To
capture heterogeneity, each simulated agent samples parameters from the
Stan prior and plays one match with those parameters.

The pipeline saves: - trial-level output (\texttt{trials.csv}) for
learning curves, - match-level summaries (\texttt{matches.csv}) for
tournament outcomes, - per-player summaries (\texttt{players.csv}) for
role effects and behavioural signatures, - LOO outputs
(\texttt{loo\_dot.csv}, \texttt{loo\_comparison\_table.csv}) for model
comparison.

\section{Results: behaviour in tournament
simulations}\label{results-behaviour-in-tournament-simulations}

\subsection{Fig 0 --- Trial dynamics and the role
swap}\label{fig-0-trial-dynamics-and-the-role-swap}

\includegraphics[width=6.67in,height=\textheight,keepaspectratio]{outputs/figs/fig0_dynamics.png}

Interpretation:

This figure shows mean cumulative pay-off for player A over trials
(ribbon = ±1 SE). The dashed vertical line marks the role swap.

What this plot is for: - It reveals dynamics: does an advantage emerge
early, slowly, or only after the swap? - It shows adaptation costs: if a
strategy needs reinforcement to re-tune after the swap, you often see a
transient kink or change in slope around the swap. - It separates
``overall win'' from ``how the win was achieved''.

\subsection{Fig 1 --- Tournament performance (who beats
whom?)}\label{fig-1-tournament-performance-who-beats-whom}

\includegraphics[width=6in,height=\textheight,keepaspectratio]{outputs/figs/fig1_matchups.png}

To make the figure interpretable as evidence, here are the exact mean
payoffs and bootstrap 95\% intervals (same summary the bars represent):

\begin{longtable}[]{@{}
  >{\raggedright\arraybackslash}p{(\linewidth - 12\tabcolsep) * \real{0.1382}}
  >{\raggedleft\arraybackslash}p{(\linewidth - 12\tabcolsep) * \real{0.1138}}
  >{\raggedleft\arraybackslash}p{(\linewidth - 12\tabcolsep) * \real{0.0569}}
  >{\raggedleft\arraybackslash}p{(\linewidth - 12\tabcolsep) * \real{0.0650}}
  >{\raggedleft\arraybackslash}p{(\linewidth - 12\tabcolsep) * \real{0.2033}}
  >{\raggedleft\arraybackslash}p{(\linewidth - 12\tabcolsep) * \real{0.2114}}
  >{\raggedleft\arraybackslash}p{(\linewidth - 12\tabcolsep) * \real{0.2114}}@{}}
\caption{Tournament outcomes with bootstrap 95\% intervals. Payoffs are
for player A.}\tabularnewline
\toprule\noalign{}
\begin{minipage}[b]{\linewidth}\raggedright
Pairing (A vs B)
\end{minipage} & \begin{minipage}[b]{\linewidth}\raggedleft
Mean payoff A
\end{minipage} & \begin{minipage}[b]{\linewidth}\raggedleft
CI low
\end{minipage} & \begin{minipage}[b]{\linewidth}\raggedleft
CI high
\end{minipage} & \begin{minipage}[b]{\linewidth}\raggedleft
Mean payoff (first half)
\end{minipage} & \begin{minipage}[b]{\linewidth}\raggedleft
Mean payoff (second half)
\end{minipage} & \begin{minipage}[b]{\linewidth}\raggedleft
Mean swing (second-first)
\end{minipage} \\
\midrule\noalign{}
\endfirsthead
\toprule\noalign{}
\begin{minipage}[b]{\linewidth}\raggedright
Pairing (A vs B)
\end{minipage} & \begin{minipage}[b]{\linewidth}\raggedleft
Mean payoff A
\end{minipage} & \begin{minipage}[b]{\linewidth}\raggedleft
CI low
\end{minipage} & \begin{minipage}[b]{\linewidth}\raggedleft
CI high
\end{minipage} & \begin{minipage}[b]{\linewidth}\raggedleft
Mean payoff (first half)
\end{minipage} & \begin{minipage}[b]{\linewidth}\raggedleft
Mean payoff (second half)
\end{minipage} & \begin{minipage}[b]{\linewidth}\raggedleft
Mean swing (second-first)
\end{minipage} \\
\midrule\noalign{}
\endhead
\bottomrule\noalign{}
\endlastfoot
kToM vs WSLS & 6.14 & 5.35 & 6.93 & 3.20 & 2.94 & -0.25 \\
WSLS vs WSLS & 0.13 & -0.67 & 0.91 & -0.06 & 0.19 & 0.26 \\
kToM vs kToM & -0.25 & -0.96 & 0.43 & -0.32 & 0.07 & 0.39 \\
WSLS vs kToM & -5.37 & -6.18 & -4.57 & -2.40 & -2.97 & -0.56 \\
\end{longtable}

Interpretation:

This is the headline ``who exploits whom?'' result. Mean payoff near 0
is what you expect when neither side has a stable advantage under
bounded rationality and noise. Large positive or negative means indicate
systematic exploitability.

The cleanest asymmetry is in the cross-play between the two different
strategies (kToM vs WSLS vs WSLS vs kToM). That is theoretically
expected: a belief learner can exploit predictable reinforcement
patterns, while WSLS does not represent the opponent.

The ``swing'' column (second half minus first half) is a direct way to
see whether the role swap changes who is advantaged within the same
match, i.e., whether the strategy interaction depends on the current
contingency.

\subsection{Fig 2 --- Role sensitivity (Matcher
advantage)}\label{fig-2-role-sensitivity-matcher-advantage}

\includegraphics[width=4.67in,height=\textheight,keepaspectratio]{outputs/figs/fig2_role_advantage.png}

A compact numeric summary:

\begin{longtable}[]{@{}lrrr@{}}
\caption{Role sensitivity: payoff as Matcher minus payoff as Mismatcher
(bootstrap 95\% CI).}\tabularnewline
\toprule\noalign{}
Strategy & Mean matcher advantage & CI low & CI high \\
\midrule\noalign{}
\endfirsthead
\toprule\noalign{}
Strategy & Mean matcher advantage & CI low & CI high \\
\midrule\noalign{}
\endhead
\bottomrule\noalign{}
\endlastfoot
WSLS & 0.076 & -0.315 & 0.473 \\
kToM & 0.011 & -0.343 & 0.377 \\
\end{longtable}

Interpretation:

This figure isolates the role manipulation. If a model represents the
goal explicitly (match vs mismatch), you expect clearer and more
systematic re-mapping across the swap. If a model is purely
reinforcement-driven, role effects can be noisier and depend on how
quickly the feedback loop re-stabilises.

In these simulations, both strategies show wide uncertainty around the
role effect. That is not surprising: Matching Pennies is balanced in
expectation, and with lapses/noise, the ``role advantage'' can be small
relative to match-to-match variability.

\subsection{Behavioural signature check: does WSLS actually look like
WSLS?}\label{behavioural-signature-check-does-wsls-actually-look-like-wsls}

A useful mechanistic sanity check is whether agents behave in a way that
matches the verbal model description. WSLS should repeat more after wins
than after losses.

\begin{longtable}[]{@{}lrr@{}}
\caption{Behavioural signature: repeat probabilities estimated from
transition counts.}\tabularnewline
\toprule\noalign{}
Strategy & Repeat after win & Repeat after loss \\
\midrule\noalign{}
\endfirsthead
\toprule\noalign{}
Strategy & Repeat after win & Repeat after loss \\
\midrule\noalign{}
\endhead
\bottomrule\noalign{}
\endlastfoot
WSLS & 0.572 & 0.556 \\
kToM & 0.711 & 0.309 \\
\end{longtable}

Interpretation:

If WSLS is implemented correctly and sampled from sensible priors, it
should show a stronger ``repeat after win'' tendency than ``repeat after
loss''. The belief learner does not encode WSLS directly, so its
repeat/shift pattern can look different and can vary with (\alpha,
\beta), and lapse.

\subsection{Fig 4 --- Payoff distributions (variance, not only means)
(optional)}\label{fig-4-payoff-distributions-variance-not-only-means-optional}

\includegraphics[width=6in,height=\textheight,keepaspectratio]{outputs/figs/fig4_payoff_distribution.png}

Interpretation:

The bar plot compresses each pairing into a mean ± interval. The violin
plot shows the full distribution of outcomes across simulations. This
matters because two strategies can have similar mean performance but
different risk profiles (stable vs high-variance). In cognitive terms,
high variance can reflect sensitivity to early random events and lapses,
not just ``skill''.

\section{Model fitting: can we identify the
mechanisms?}\label{model-fitting-can-we-identify-the-mechanisms}

A mechanistic model should not only generate behaviour; it should also
be distinguishable when fit to data. Here we fit both models to
synthetic datasets generated by each strategy and compare them with
approximate leave-one-out cross-validation (LOO).

\subsection{Fig 5 --- LOO model
comparison}\label{fig-5-loo-model-comparison}

\includegraphics[width=4.67in,height=\textheight,keepaspectratio]{outputs/figs/fig5_loo_dot.png}

Key numbers (ELPD difference = Belief − WSLS; bars are ±2 SE):

\begin{itemize}
\tightlist
\item
  WSLS-generated data: (\Delta\text{ELPD} =) -31.43 (SE = 8.27, approx
  95\% range {[}-47.97, -14.9{]}).
\item
  Belief-generated data: (\Delta\text{ELPD} =) 11.02 (SE = 26.82, approx
  95\% range {[}-42.61, 64.66{]}).
\end{itemize}

Interpretation:

Positive differences mean the Belief model predicts better; negative
means WSLS predicts better.

The desirable pattern is ``match the generator'': WSLS should win on
WSLS-generated data, and Belief should win on Belief-generated data.
When the interval includes 0, it indicates that the data horizon/noise
makes discrimination difficult (a real identifiability issue rather than
a coding bug).

\subsection{Fig 3 --- Posterior parameter interpretation (example
subjects)}\label{fig-3-posterior-parameter-interpretation-example-subjects}

\includegraphics[width=6.67in,height=\textheight,keepaspectratio]{outputs/figs/fig3_interpretation_combined.png}

Interpretation:

Posterior densities show how the data constrain the cognitive parameters
(learning rate, exploitation strength, perseveration, lapse). The dashed
line is the true generating value (parameter recovery check).

In Matching Pennies, perfect recovery is not expected in general: the
equilibrium behaviour can look close to random, and with limited trials
+ lapse noise, multiple parameter settings can generate similar
sequences. The correct conclusion is therefore about which parameters
are identifiable under the current design and how uncertainty reflects
cognitive ambiguity in the task.

\section{Discussion (why these results make
sense)}\label{discussion-why-these-results-make-sense}

WSLS and kToM-inspired belief learning make different cognitive
commitments.

WSLS is a bounded-memory reinforcement heuristic. It is plausible as a
``cheap'' strategy but it is systematically exploitable by opponents
that can detect its regularities. It also has no explicit representation
of the role contingency; it adapts only through feedback.

The belief learner adds a small internal state (belief about the
opponent) and explicit conditioning on the match/mismatch goal. This
makes it more flexible around the role swap and better suited to
exploiting predictable opponents, while still being cognitively
constrained (compressed memory, incremental update, lapse noise).

The core figures connect these commitments to observable signatures: (i)
trial dynamics and swap adaptation, (ii) robust pairing differences
across simulations, (iii) whether role matters systematically, and (iv)
whether the two models are distinguishable in model comparison.

\section{Reproducibility (how to run)}\label{reproducibility-how-to-run}

From the \texttt{assignment1/} folder:

\begin{Shaded}
\begin{Highlighting}[]
\ExtensionTok{Rscript}\NormalTok{ run\_all.R}
\end{Highlighting}
\end{Shaded}

This generates the CSVs in \texttt{outputs/data/} and the figures in
\texttt{outputs/figs/}, which this document loads.




\end{document}
